\documentclass[journal]{vgtc}                % final (journal style)
%\documentclass[review,journal]{vgtc}         % review (journal style)
%\documentclass[widereview]{vgtc}             % wide-spaced review
%\documentclass[preprint,journal]{vgtc}       % preprint (journal style)

%% Uncomment one of the lines above depending on where your paper is
%% in the conference process. ``review'' and ``widereview'' are for review
%% submission, ``preprint'' is for pre-publication, and the final version
%% doesn't use a specific qualifier.

%% Please use one of the ``review'' options in combination with the
%% assigned online id (see below) ONLY if your paper uses a double blind
%% review process. Some conferences, like IEEE Vis and InfoVis, have NOT
%% in the past.

%% Please note that the use of figures other than the optional teaser is not permitted on the first page
%% of the journal version.  Figures should begin on the second page and be
%% in CMYK or Grey scale format, otherwise, colour shifting may occur
%% during the printing process.  Papers submitted with figures other than the optional teaser on the
%% first page will be refused. Also, the teaser figure should only have the
%% width of the abstract as the template enforces it.

%% These few lines make a distinction between latex and pdflatex calls and they
%% bring in essential packages for graphics and font handling.
%% Note that due to the \DeclareGraphicsExtensions{} call it is no longer necessary
%% to provide the the path and extension of a graphics file:
%% \includegraphics{diamondrule} is completely sufficient.
%%
\ifpdf%                                % if we use pdflatex
  \pdfoutput=1\relax                   % create PDFs from pdfLaTeX
  \pdfcompresslevel=9                  % PDF Compression
  \pdfoptionpdfminorversion=7          % create PDF 1.7
  \ExecuteOptions{pdftex}
  \usepackage{graphicx}                % allow us to embed graphics files
  \DeclareGraphicsExtensions{.pdf,.png,.jpg,.jpeg} % for pdflatex we expect .pdf, .png, or .jpg files
\else%                                 % else we use pure latex
  \ExecuteOptions{dvips}
  \usepackage{graphicx}                % allow us to embed graphics files
  \DeclareGraphicsExtensions{.eps}     % for pure latex we expect eps files
\fi%

%% it is recomended to use ``\autoref{sec:bla}'' instead of ``Fig.~\ref{sec:bla}''
\graphicspath{{figures/}{pictures/}{images/}{./}} % where to search for the images

\usepackage{microtype}                 % use micro-typography (slightly more compact, better to read)
\PassOptionsToPackage{warn}{textcomp}  % to address font issues with \textrightarrow
\usepackage{textcomp}                  % use better special symbols
\usepackage{mathptmx}                  % use matching math font
\usepackage{times}                     % we use Times as the main font
\renewcommand*\ttdefault{txtt}         % a nicer typewriter font
\usepackage{cite}                      % needed to automatically sort the references
\usepackage{tabu}                      % only used for the table example
\usepackage{booktabs}                  % only used for the table example
%% We encourage the use of mathptmx for consistent usage of times font
%% throughout the proceedings. However, if you encounter conflicts
%% with other math-related packages, you may want to disable it.

%% In preprint mode you may define your own headline.
%\preprinttext{To appear in IEEE Transactions on Visualization and Computer Graphics.}

%% If you are submitting a paper to a conference for review with a double
%% blind reviewing process, please replace the value ``0'' below with your
%% OnlineID. Otherwise, you may safely leave it at ``0''.
\onlineid{0}

%% declare the category of your paper, only shown in review mode
\vgtccategory{Research}
%% please declare the paper type of your paper to help reviewers, only shown in review mode
%% choices:
%% * algorithm/technique
%% * application/design study
%% * evaluation
%% * system
%% * theory/model
\vgtcpapertype{please specify}

%% Paper title.
\title{Instagram Visualization - Assignment 1}

%% This is how authors are specified in the journal style

%% indicate IEEE Member or Student Member in form indicated below
\author{Courtney Bonn and Zach Sherman}
\authorfooter{
%% insert punctuation at end of each item
\item
 Courtney Bonn is with Oregon State University. E-mail: bonnc@oregonstate.edu.
\item
 Zach Sherman is with Oregon State University. E-mail: shermaza@oregonstate.edu.

}

%other entries to be set up for journal
\shortauthortitle{Biv \MakeLowercase{\textit{et al.}}: Global Illumination for Fun and Profit}
%\shortauthortitle{Firstauthor \MakeLowercase{\textit{et al.}}: Paper Title}

%% Abstract section.
\abstract{Instagram's API service provides access to interesting information about the relationships between users and their media. A visualization of this information would be beneficial to users who would like to get a "big picture" idea of how other users are interacting with their profile. The visualization would primarily help users discover different demographics within their follower base, but it could also help see what kinds of photos generated new users, and whether or not certain subgroups of users tend to interact more with certain kinds of media. We believe this visualization could be built utilizing modern web technologies.%
} % end of abstract

%% Keywords that describe your work. Will show as 'Index Terms' in journal
%% please capitalize first letter and insert punctuation after last keyword
\keywords{Visualization, Instagram}

%% ACM Computing Classification System (CCS). 
%% See <http://www.acm.org/class/1998/> for details.
%% The ``\CCScat'' command takes four arguments.

\CCScatlist{ % not used in journal version
 \CCScat{K.6.1}{Management of Computing and Information Systems}%
{Project and People Management}{Life Cycle};
 \CCScat{K.7.m}{The Computing Profession}{Miscellaneous}{Ethics}
}

%% Uncomment below to include a teaser figure.
%%\teaser{
%% \centering
%%  \includegraphics[width=\linewidth]{CypressView}
%%  \caption{In the Clouds: Vancouver from Cypress Mountain. Note that the teaser may not be wider than the abstract block.}
%%	\label{fig:teaser}
%%}

%% Uncomment below to disable the manuscript note
\renewcommand{\manuscriptnotetxt}{}

%% Copyright space is enabled by default as required by guidelines.
%% It is disabled by the 'review' option or via the following command:
% \nocopyrightspace

\vgtcinsertpkg

%%%%%%%%%%%%%%%%%%%%%%%%%%%%%%%%%%%%%%%%%%%%%%%%%%%%%%%%%%%%%%%%
%%%%%%%%%%%%%%%%%%%%%% START OF THE PAPER %%%%%%%%%%%%%%%%%%%%%%
%%%%%%%%%%%%%%%%%%%%%%%%%%%%%%%%%%%%%%%%%%%%%%%%%%%%%%%%%%%%%%%%%

\begin{document}

%% The ``\maketitle'' command must be the first command after the
%% ``\begin{document}'' command. It prepares and prints the title block.

%% the only exception to this rule is the \firstsection command
\firstsection{Introduction}

\maketitle

%% \section{Introduction} %for journal use above \firstsection{..} instead
Instagram began as a simple photo sharing application, but it has grown into a social media giant. Companies use it to market their products, individuals use it to tell stories and share what they enjoy, and groups of individuals can use it as a platform to share pictures privately. 

On top of the media, the application allows users to like and comment on photos, and the data underneath that system can shed some insight on a profile is being interacted with, which can be used to improve the profile's effectiveness. This data is not easily visualized mentally, due primarily to the sheer size of the relationships.

Users who want to see how other users are interacting with their profile would benefit from a visualization of this information. We suggest that a visualization utilizing spacial recognition and relative size can effectively portray this information in a way that is beneficial to the previously mentioned users.

\section{Visualization Tasks}

This visualization is intended to solve problems regarding the relationship between the primary user and users who have interacted with their profile. These problems include:
\begin{itemize}
\item How can I tell if there are multiple demographics interacting with my profile?
\item Is there a certain subset of users that seem to like specific photos?
\item Which of my photos helped to generate new followers?
\end{itemize}


\section{Data Sources}
The primary resource for fetching data comes from the Instagram API, which is described by the official documentation. From this API, we have access to endpoints with data about a user's profile, their media, and attributes about that media such as who liked the media and who commented on it ("Instagram Developer Documentation").

\section{ER Diagrams}
\includegraphics[width=\linewidth]{erd}

\section{Implementation of ER diagrams}
The data itself is already implemented in Instagram's API service. What we suggest is to pull from the API dynamically, formatting the data during runtime for the visualization.

\section{Visualization Interface Design}
The visualization itself is going to look similar to a force-directed graph with two kinds of nodes. The primary nodes represent a Media object. These are photos or videos, and they will display the photo or a screen shot of the video. Secondary nodes will represent User populations related to primary nodes, displaying the number of users being represented in that node.

If one user has liked a media object, a secondary node is created and attached to the media node. As the number of users increases who liked the same media object, the scale of the secondary node increases.

The interface will include one secondary node for every combination of users and specific photos they liked or commented on. For example, two users who liked two of the three media nodes will get a separate node from a user who has interacted with all three media nodes.



% %% if specified like this the section will be committed in review mode
% \acknowledgments{
% The authors wish to thank A, B, C. This work was supported in part by
% a grant from XYZ.}

%\bibliographystyle{abbrv}
\bibliographystyle{abbrv-doi}
%\bibliographystyle{abbrv-doi-narrow}
%\bibliographystyle{abbrv-doi-hyperref}
%\bibliographystyle{abbrv-doi-hyperref-narrow}

\bibliography{template}
"Instagram Developer Documentation". Instagram.com. N.p., 2017. Web. 16 Apr. 2017.
\end{document}

